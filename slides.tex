\documentclass[compress]{beamer}

\beamertemplatetransparentcovered
\setbeamertemplate{navigation symbols}{}
\setbeamertemplate{footline}
{\leavevmode%
  \hbox{%
    \begin{beamercolorbox}[wd=.3333\paperwidth,left]{}%
      \insertshortinstitute
    \end{beamercolorbox}%
    \begin{beamercolorbox}[wd=.3333\paperwidth,center]{}%
      \insertframenumber{} / \inserttotalframenumber
    \end{beamercolorbox}
    \begin{beamercolorbox}[wd=.3333\paperwidth,right]{}%
      \insertshortdate{}
      \;
    \end{beamercolorbox}
  }%
}

\usepackage{listings}
\usepackage{color}
\usepackage{minted}
\usepackage{hyperref}
\usepackage{textcomp}
\usepackage{marvosym}
\usepackage{xcolor}
\usepackage{graphicx}

\usetheme[light,accent=violet]{solarized}

\usepackage{fontspec}
\setmainfont[Mapping=tex-text]{Source Sans Pro}
\setmonofont[Mapping=tex-text]{Inconsolata}

\graphicspath{{./images/}}

\title{On Rails Engines.}
\subtitle{How to profit, test and deploy.}

\author{Stas Sușcov (\href{mailto:stas@nerd.ro}{stas@nerd.ro})}
\date{January 26th, 2013}

\institute{GeekMeet \#14, Cluj-Napoca, Transylvania}

\begin{document}

% ----------------------------------------------------------------------------
% *** Titlepage <<<
% ----------------------------------------------------------------------------
\maketitle
% ----------------------------------------------------------------------------
% *** END of Titlepage >>>
% ----------------------------------------------------------------------------

% ----------------------------------------------------------------------------
% *** Frame <<<
% ----------------------------------------------------------------------------
\begin{frame}
\frametitle{About Stas}

\begin{itemize}
  \item a \href{http://stas.nerd.ro}{nerd}
  \item picky developer
  \item interests: web/operations
  \item on Ruby since 2011
  \item (\Heart (food wine cycling))
\end{itemize}
\end{frame}
% ----------------------------------------------------------------------------
% *** END of Frame >>>
% ----------------------------------------------------------------------------

% ----------------------------------------------------------------------------
% *** Frame <<<
% ----------------------------------------------------------------------------
\begin{frame}
\frametitle{Selfish self-promotion}
  \begin{center}
    \huge We started \emph{Coursewa.re} in 2012, which is a SaaS social learning environment.
    \\
    \small Lots of experience I am sharing today comes from building this platform.
    \colorbox{solarizedBase3}{\includegraphics[height=2cm]{courseware.png}}
  \end{center}
\end{frame}
% ----------------------------------------------------------------------------
% *** END of Frame >>>
% ----------------------------------------------------------------------------

% ----------------------------------------------------------------------------
% *** Frame <<<
% ----------------------------------------------------------------------------
\begin{frame}
\frametitle{The story}

\begin{itemize}[<+->]
  \item started as a Rails project
  \item business requirements started to change very quickly
  \item project use-cases started to vary
  \item maintenance time started to grow (tests, updates \& such)
  \item development got slower (the bigger the codebase, the noisier it gets)
\end{itemize}
\end{frame}
% ----------------------------------------------------------------------------
% *** END of Frame >>>
% ----------------------------------------------------------------------------

% ----------------------------------------------------------------------------
% *** Frame <<<
% ----------------------------------------------------------------------------
\begin{frame}
\frametitle{Identify the issue}

\begin{itemize}[<+->]
  \item stop looking at your code, that is \emph{not} your problem, here is why?!
    \begin{itemize}
      \item unless you do not have plenty of cash, you will spend lots of resources to win some speed
      \item unless you do not have time constraints, doing some refactoring can help
      \item unless you love refactoring, you will want to work on features rather rewriting components/tests
    \end{itemize}
  \item look at the code organization
    \begin{itemize}
      \item because running the \emph{test suite} for 30 minutes is not fun!
      \item because your CI will love fast builds
      \item because your backed developers do not care about pull requests related to assets or views
      \item because at some point you will want an API for your app, and some OAuth, and that better be not the same old controller
    \end{itemize}
  \item none of the above?! then I am wrong and you should look at your code
\end{itemize}
\end{frame}
% ----------------------------------------------------------------------------
% *** END of Frame >>>
% ----------------------------------------------------------------------------

% ----------------------------------------------------------------------------
% *** Frame <<<
% ----------------------------------------------------------------------------
\begin{frame}
  \begin{center}
    \huge My point is that building a modular/components-based codebase is faster, cheaper and easier.
    \\
    \small Sure it needs some know-how, but hey, you were not born programming, you learned that too!
  \end{center}
\end{frame}
% ----------------------------------------------------------------------------
% *** END of Frame >>>
% ----------------------------------------------------------------------------

% ----------------------------------------------------------------------------
% *** Frame <<<
% ----------------------------------------------------------------------------
\begin{frame}
\frametitle{Looking at Rails Engines}

\begin{itemize}[<+->]
  \item pros
    \begin{itemize}
      \item every Rails app is basically an engine, which ensures no integration issues
        \begin{itemize}
          \item migrations will run fine
          \item mount it and leverage from any defined routes
          \item inherit any engine resource to customize/change the behaviour
        \end{itemize}
      \item all your Rails gems will play nice with engines
      \item Bundle it in your Gemfile, it is just another gem in the end
      \item engine has its own tests, reduces testing time
      \item codebase gets smaller, easier to follow / assign across teams
    \end{itemize}
  \item cons
    \begin{itemize}
      \item testing engines might not be so straight
      \item engines are not yet first class Rails citizens, get ready to patch things along the way
      \item the \emph{dummy app} issue
      \item not very popular, you get better at engines by working with engines
    \end{itemize}
\end{itemize}
\end{frame}
% ----------------------------------------------------------------------------
% *** END of Frame >>>
% ----------------------------------------------------------------------------

% ----------------------------------------------------------------------------
% *** Frame <<<
% ----------------------------------------------------------------------------
\begin{frame}
  \begin{center}
    \huge Overall: you will love it, unless your app complexity/functionality is not the subject.
  \end{center}
\end{frame}
% ----------------------------------------------------------------------------
% *** END of Frame >>>
% ----------------------------------------------------------------------------

% ----------------------------------------------------------------------------
% *** Frame <<<
% ----------------------------------------------------------------------------
\begin{frame}[fragile]
\frametitle{Creating an engine}

\begin{center}
  \begin{tiny}
    There are two types of engines:
    \\
    \textbf{Left:} mountable, engine close to a full app, has routes, controllers, views and assets
    \\
    \textbf{Right:} simple, close to a generic gem structure; ideal to build Rails specific gems (generators, assets, etc\ldots)
    \\
    \textbf{Tip:} avoid characters like \texttt{-} or \texttt{\_} in engine names.
  \end{tiny}
\end{center}

\begin{columns}
  \begin{column}{0.6\textwidth}
    \begin{minted}[fontsize=\tiny,gobble=6]{bash}
      ~$ rails plugin new gmengine --skip-test-unit --mountable
      create  README.rdoc
      create  Rakefile
      create  gmengine.gemspec
      create  MIT-LICENSE
      create  Gemfile
      create  app
      create  app/controllers/gmengine/application_controller.rb
      create  app/helpers/gmengine/application_helper.rb
      create  app/views/layouts/gmengine/application.html.erb
      create  app/assets/images/gmengine
      create  config/routes.rb
      create  lib/gmengine.rb
      create  lib/tasks/gmengine_tasks.rake
      create  lib/gmengine/version.rb
      create  lib/gmengine/engine.rb
      create  app/assets/stylesheets/gmengine/application.css
      create  app/assets/javascripts/gmengine/application.js
      create  script
      create  script/rails
         run  bundle install
    \end{minted}
  \end{column}

  \begin{column}{0.4\textwidth}
    \begin{minted}[fontsize=\tiny,gobble=6]{bash}
      ~$ rails plugin new gmengine --skip-test-unit
      create  README.rdoc
      create  Rakefile
      create  gmengine.gemspec
      create  MIT-LICENSE
      create  Gemfile
      create  lib/gmengine.rb
      create  lib/tasks/gmengine_tasks.rake
      create  lib/gmengine/version.rb
         run  bundle install
    \end{minted}
    \end{column}
  \end{columns}
\end{frame}
% ----------------------------------------------------------------------------
% *** END of Frame >>>
% ----------------------------------------------------------------------------

% ----------------------------------------------------------------------------
% *** Frame <<<
% ----------------------------------------------------------------------------
\begin{frame}
\frametitle{A closer look at the mountable engine}

There are two important files that make an engine gem look different from a regular one.
\begin{itemize}
    \begin{item}
      \texttt{gmengine/lib/gmengine.rb}
      \inputminted[fontsize=\tiny,gobble=0,linenos=true,firstline=0,lastline=0]{ruby}{code/gmengine/lib/gmengine.rb}
    \end{item}
    \begin{item}
      \texttt{gmengine/lib/gmengine/engine.rb}
      \inputminted[fontsize=\tiny,gobble=0,linenos=true,firstline=0,lastline=0]{ruby}{code/gmengine/lib/gmengine/engine.rb}
    \end{item}
\end{itemize}
\end{frame}
% ----------------------------------------------------------------------------
% *** END of Frame >>>
% ----------------------------------------------------------------------------

% ----------------------------------------------------------------------------
% *** Frame <<<
% ----------------------------------------------------------------------------
\begin{frame}[fragile]
\frametitle{Working with an engine}

There is not much difference from creating controllers or models for an engine
than for a regular Rails app.
\begin{itemize}
    \begin{item}
      Generating a model
      \begin{minted}[fontsize=\tiny,gobble=6]{bash}
        ~$ rails g model post title:string content:text
        invoke  active_record
        create    db/migrate/20130120140937_create_gmengine_posts.rb
        create    app/models/gmengine/post.rb
      \end{minted}
    \end{item}
    \begin{item}
      Generating a controller
      \begin{minted}[fontsize=\tiny,gobble=6]{bash}
        ~$ rails g controller posts new index
        create  app/controllers/gmengine/posts_controller.rb
        route  get "posts/index"
        route  get "posts/new"
        invoke  erb
        create    app/views/gmengine/posts
        create    app/views/gmengine/posts/new.html.erb
        create    app/views/gmengine/posts/index.html.erb
        invoke  helper
        create    app/helpers/gmengine/posts_helper.rb
        invoke    rspec
        create      spec/helpers/gmengine/posts_helper_spec.rb
        invoke  assets
        invoke    js
        create      app/assets/javascripts/gmengine/posts.js
        invoke    css
        create      app/assets/stylesheets/gmengine/posts.css
      \end{minted}
    \end{item}
\end{itemize}
\end{frame}
% ----------------------------------------------------------------------------
% *** END of Frame >>>
% ----------------------------------------------------------------------------

% ----------------------------------------------------------------------------
% *** Frame <<<
% ----------------------------------------------------------------------------
\begin{frame}
\frametitle{Testing a Rails engine}

A Rails engine requires a separate \emph{dummy} app in order to be tested.
There are two options here:
\begin{itemize}
  \item generate the app and carry it over in your \textbf{spec} directory
  \item write a generator that will bootstrap an app before running tests
\end{itemize}

The first option is not that \emph{DRY} and (personal opinion) conceptually wrong.
Until Rails v4.0 is released, here is an example of such a generator.
\end{frame}
% ----------------------------------------------------------------------------
% *** END of Frame >>>
% ----------------------------------------------------------------------------

% ----------------------------------------------------------------------------
% *** Frame <<<
% ----------------------------------------------------------------------------
\begin{frame}
\frametitle{Rails dummy app generator}

\texttt{gmengine/lib/gmengine/generators/dummy\_generator.rb}
\inputminted[fontsize=\tiny,gobble=0,linenos=true,firstline=0,lastline=0]{ruby}{code/gmengine/lib/gmengine/generators/dummy_generator.rb}
\end{frame}
% ----------------------------------------------------------------------------
% *** END of Frame >>>
% ----------------------------------------------------------------------------

% ----------------------------------------------------------------------------
% *** Frame <<<
% ----------------------------------------------------------------------------
\begin{frame}
\frametitle{Preparing the engine for testing}

We will be using RSpec. Here are the steps in order to get things ready:
\begin{itemize}
  \item add \texttt{rspec-rails} to your \textbf{gemspec} file
    \begin{item}
      customize the task to run our dummy app generator before starting any test
      (see the next slide)
    \end{item}
  \item install RSpec \texttt{rails generate rspec:install}
  \begin{item}
    tweak \texttt{spec/spec\_helper.rb} to load the dummy app for tests
    \inputminted[fontsize=\tiny,gobble=0,linenos=true,firstline=3,lastline=7]{ruby}{code/gmengine/spec/spec_helper.rb}
  \end{item}
\end{itemize}
\end{frame}
% ----------------------------------------------------------------------------
% *** END of Frame >>>
% ----------------------------------------------------------------------------

% ----------------------------------------------------------------------------
% *** Frame <<<
% ----------------------------------------------------------------------------
\begin{frame}
  \texttt{gmengine/Rakefile}
  \inputminted[fontsize=\tiny,gobble=0,linenos=true,firstline=26,lastline=0]{ruby}{code/gmengine/Rakefile}
\end{frame}
% ----------------------------------------------------------------------------
% *** END of Frame >>>
% ----------------------------------------------------------------------------

% ----------------------------------------------------------------------------
% *** Frame <<<
% ----------------------------------------------------------------------------
\begin{frame}
\frametitle{Testing routes / controllers}
  Make sure you always pass \texttt{:use\_route} with your engine name value.
  \\
  \small\texttt{gmengine/spec/controllers/gmengine/post\_controller\_spec.rb}
  \inputminted[fontsize=\tiny,gobble=0,linenos=true,firstline=0,lastline=0]{ruby}{code/gmengine/spec/controllers/gmengine/posts_controller_spec.rb}
\end{frame}
% ----------------------------------------------------------------------------
% *** END of Frame >>>
% ----------------------------------------------------------------------------

% ----------------------------------------------------------------------------
% *** Frame <<<
% ----------------------------------------------------------------------------
\begin{frame}[fragile]
\frametitle{Using a Rails engine}
Engines are pretty much gems, just update your \texttt{Gemfile} to start using it.

\begin{minted}[fontsize=\tiny,gobble=2]{bash}
  ~$ cat Gemfile | ack gmengine
  gem 'gmengine', :path => 'path/to/gmengine'
\end{minted}
\end{frame}
% ----------------------------------------------------------------------------
% *** END of Frame >>>
% ----------------------------------------------------------------------------

% ----------------------------------------------------------------------------
% *** Frame <<<
% ----------------------------------------------------------------------------
\begin{frame}
  \begin{center}
  \huge Questions please\ldots
  \\
  Thank you for your time.
  \end{center}
\end{frame}
% ----------------------------------------------------------------------------
% *** END of Frame >>>
% ----------------------------------------------------------------------------

% ----------------------------------------------------------------------------
% *** Frame <<<
% ----------------------------------------------------------------------------
\begin{frame}
\frametitle{Online resources worth checking}

\begin{itemize}
  \item Rails Engine Docs
    \begin{itemize}
      \item Guide -- \href{http://guides.rubyonrails.org/engines.html}{guides.rubyonrails.org/engines.html}
      \item API -- \href{http://edgeapi.rubyonrails.org/classes/Rails/Engine.html}{edgeapi.rubyonrails.org/classes/Rails/Engine.html}
    \end{itemize}
  \item Writing a Rails Engine by Erik Michael-Ober -- \href{http://www.youtube.com/watch?v=Rvxcc46fox0}{youtube.com/watch?v=Rvxcc46fox0}
  \item Rails Engines in real world
    \begin{itemize}
      \item Forem gem -- \href{https://github.com/radar/forem}{github.com/radar/forem}
      \item Spree gem -- \href{https://github.com/spree}{github.com/spree}
      \item Travis-CI -- \href{https://github.com/travis-ci}{github.com/travis-ci}
    \end{itemize}
  \item These slides -- \href{https://github.com/stas/rails-engines-slides-geekmeet}{github.com/stas/rails-engines-slides-geekmeet}
\end{itemize}

\end{frame}
% ----------------------------------------------------------------------------
% *** END of Frame >>>
% ----------------------------------------------------------------------------

\end{document}
